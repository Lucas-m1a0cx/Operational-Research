\documentclass{article}

\usepackage[UTF8]{ctex}
\usepackage{placeins}
\usepackage{graphicx}
\usepackage{listings}
\usepackage{xcolor} % 添加 xcolor 宏包
\usepackage{amsmath}
\usepackage{amssymb}
\usepackage{array}
\lstset{ %
language=matlab,                % choose the language of the code
basicstyle=\footnotesize,       % the size of the fonts that are used for the code
numbers=left,                   % where to put the line-numbers
numberstyle=\footnotesize,      % the size of the fonts that are used for the line-numbers
stepnumber=1,                   % the step between two line-numbers. If it is 1 each line will be numbered
numbersep=5pt,                  % how far the line-numbers are from the code
backgroundcolor=\color{white},  % choose the background color. You must add \usepackage{color}
showspaces=false,               % show spaces adding particular underscores
showstringspaces=false,         % underline spaces within strings
showtabs=false,                 % show tabs within strings adding particular underscores
frame=single,           % adds a frame around the code
tabsize=2,          % sets default tabsize to 2 spaces
captionpos=b,           % sets the caption-position to bottom
breaklines=true,        % sets automatic line breaking
breakatwhitespace=false,    % sets if automatic breaks should only happen at whitespace
escapeinside={\%*}{*)}          % if you want to add a comment within your code
}
\title{hw01}


\begin{document}
\maketitle
    \section*{Q1(1)}
    自拟数据如下:
    \begin{table}[h]
      \centering
      \begin{tabular}{|c|c|c|c|c|}
      \hline
       & A1 & A2 & A3 & 营养成分最低需求量 \\
      \hline
      B1 & 2 & 3 & 1 & 15 \\
      \hline
      B2 & 4 & 1 & 2 & 10 \\
      \hline
      食品单价 & 3 & 2 & 1 &  \\
      \hline
      \end{tabular}
      \caption{数据表格}
  \end{table}
    \begin{lstlisting}[caption={题1(1) MATLAB代码}, label={lst:matlab}]
        n = 3; % 食品种类数
        m = 2; % 营养成分种类数
        a = [2 3 1; 4 1 2]; % 含量矩阵
        b = [15; 10]; % 最低需求量向量
        c = [3; 2; 1]; % 单价向量

        % 定义优化问题
        f = c; % 目标函数系数向量
        A = -a; % 不等式约束左侧矩阵
        b_ub = -b; % 不等式约束右侧向量
        lb = zeros(n, 1); % 变量下界向量
        intcon = 1:n; % 整数变量索引向量

        % 求解优化问题
        [x, fval] = intlinprog(f, intcon, A, b_ub, [], [], lb, []);
        disp('最低花费:');
        disp(fval);
        disp('食品份数:');
        disp(x);

    \end{lstlisting}
    Answer:
      \[\text{最低花费:}11\]
      \[\text{食品份数:}0, 4, 3\]
    \section*{Q1(2)}
      \[d_1=5, d_2=6, d_3=4\]
      \begin{lstlisting}[caption={题1(1) MATLAB代码}, label={lst:matlab}]
        % 手动输入数据
        n = 3; % 食品种类数
        m = 2; % 营养成分种类数
        a = [2 3 1; 4 1 2]; % 含量矩阵
        b = [15; 10]; % 最低需求量向量
        c = [3; 2; 1]; % 单价向量
        d = [5; 6; 4]; % 最低摄入量向量

        % 定义优化问题
        f = c; % 目标函数系数向量
        A = -a; % 不等式约束左侧矩阵
        b_ub = -b; % 不等式约束右侧向量
        Aeq = ones(1, n); % 等式约束左侧矩阵
        beq = sum(d); % 等式约束右侧向量,保证总摄入量满足要求
        lb = zeros(n, 1); % 变量下界向量
        intcon = 1:n; % 整数变量索引向量

        % 求解优化问题
        [x, fval] = intlinprog(f, intcon, A, b_ub, Aeq, beq, lb, []);
        disp('最低花费:');
        disp(fval);
        disp('食品份数:');
        disp(x);

      \end{lstlisting}
    Answer:
      \[\text{最低花费:}15\]
      \[\text{食品份数:}0, 0, 15\]
    \section*{Q2}
    \noindent \textbf{解:} \\
    用\(i=1,2\)分别代表重型和轻型炸弹,\(j=1,2,3,4\)分别代表四个要害部位,\(x_{ij}\)为投到第\(j\)部位的\(i\)种型号炸弹的数量,
    则问题的数学模型可以等价为求一个目标都不能命中的最小可能值:
    \[\min z = {\left( {1 - 0.10} \right)^{{x_{11}}}} \cdot {\left( {1 - 0.20} \right)^{{x_{12}}}} \cdot {\left( {1 - 0.15} \right)^{{x_{13}}}} \cdot {\left( {1 - 0.25} \right)^{{x_{14}}}} \cdot {\left( {1 - 0.08} \right)^{{x_{21}}}} \cdot   \]
    \[{\left( {1 - 0.16} \right)^{{x_{22}}}} \cdot {\left( {1 - 0.12} \right)^{{x_{23}}}} \cdot {\left( {1 - 0.20} \right)^{{x_{24}}}}\]
    且需满足约束条件:
    \[\left\{ {\begin{array}{*{20}{c}}
        \begin{gathered}
        \frac{{1.5 \cdot 450}}{2}{x_{11}} + \frac{{1.5 \cdot 480}}{2}{x_{12}} + \frac{{1.5 \cdot 540}}{2}{x_{13}} + \frac{{1.5 \cdot 600}}{2}{x_{14}} +  \hfill \\
        \frac{{1.75 \cdot 450}}{3}{x_{21}} + \frac{{1.75 \cdot 480}}{3}{x_{22}} + \frac{{2 \cdot 540}}{3}{x_{23}} + \frac{{2 \cdot 600}}{3}{x_{24}} +  \hfill \\
        100\left( {{x_{11}} + {x_{12}} + {x_{13}} + {x_{14}} + {x_{21}} + {x_{22}} + {x_{23}} + {x_{24}}} \right) \leqslant 48000 \hfill \\ 
      \end{gathered}  \\ 
        {{x_{11}} + {x_{12}} + {x_{13}} + {x_{14}} \leqslant 32} \\ 
        {{x_{21}} + {x_{22}} + {x_{23}} + {x_{24}} \leqslant 48} \\ 
        {\begin{array}{*{20}{c}}
        {{x_{ij}} \geqslant 0}&{i = 1,2;j = 1, \cdots ,4} 
      \end{array}} 
      \end{array}} \right.\]

\end{document}